%%%%%%%%%%%%%%%%%%%%%%%%%%%%%%%%%%%%%%%%%
% fphw Assignment
% LaTeX Template
% Version 1.0 (27/04/2019)
%
% This template originates from:
% https://www.LaTeXTemplates.com
%
% Authors:
% Class by Felipe Portales-Oliva (f.portales.oliva@gmail.com) with template 
% content and modifications by Vel (vel@LaTeXTemplates.com)
%
% Template (this file) License:
% CC BY-NC-SA 3.0 (http://creativecommons.org/licenses/by-nc-sa/3.0/)
%
%%%%%%%%%%%%%%%%%%%%%%%%%%%%%%%%%%%%%%%%%

%----------------------------------------------------------------------------------------
%	PACKAGES AND OTHER DOCUMENT CONFIGURATIONS
%----------------------------------------------------------------------------------------

\documentclass[
	12pt, % Default font size, values between 10pt-12pt are allowed
	letterpaper, % Uncomment for US letter paper size
	%spanish, % Uncomment for Spanish
]{fphw}

% Template-specific packages
\usepackage[utf8]{inputenc} % Required for inputting international characters
\usepackage[T1]{fontenc} % Output font encoding for international characters
\usepackage{mathpazo} % Use the Palatino font

\usepackage{graphicx} % Required for including images

\usepackage{booktabs} % Required for better horizontal rules in tables

\usepackage{listings} % Required for insertion of code

\usepackage{enumerate} % To modify the enumerate environment

\usepackage{amsmath}

%----------------------------------------------------------------------------------------
%	ASSIGNMENT INFORMATION
%----------------------------------------------------------------------------------------

\title{Homework \#1} % Assignment title

\author{Fangzhou Ai} % Student name

\date{Jan 18th, 2021} % Due date

\institute{University of California, San Diego \\ Department of Electrical and Computer Engineering} % Institute or school name

\class{CSE 251B, WI 21} % Course or class name

\professor{Dr. Gary Cottrell} % Professor or teacher in charge of the assignment

%----------------------------------------------------------------------------------------

\begin{document}

\maketitle % Output the assignment title, created automatically using the information in the custom commands above

%----------------------------------------------------------------------------------------
%	ASSIGNMENT CONTENT
%----------------------------------------------------------------------------------------

\section*{Question 1}

\begin{problem}
	Work problems 1-4 (5 points each) on pages 28-30 of Bishop.
\end{problem}

%------------------------------------------------

\subsection*{Answer}

\begin{enumerate}[(\itshape a\normalfont)] % Sub-questions styled as italic letters
	\item From (1.41) we could derive
	\begin{equation}
	\begin{aligned}
	& \int_{-\infty}^{\infty} exp\{-\frac{\lambda}{2}x^2\}dx = (\frac{2\pi}{\lambda})^{1/2} \\
	\Rightarrow & \int_{-\infty}^{\infty} exp\{-(\sqrt{\frac{\lambda}{2}}x)^2\}d\sqrt{\frac{\lambda}{2}}x = \pi^{1/2} \\
	\Rightarrow & \int_{-\infty}^{\infty} exp\{-x^2\}dx = \pi^{1/2} \\
	\end{aligned}
	\end{equation}
	
	Therefore, the result of the L.H.S of (1.42) is $\pi ^{d/2}$, thus we can convert the 1.42 into following form
	\begin{equation}
	\begin{aligned}
	& \pi ^{d/2} = S_d \int_{0}^{\infty}e^{-r^2}r^{d-1}dr \\
	\Leftrightarrow & S_d = \frac{\pi ^{d/2}}{\int_{0}^{\infty}e^{-r^2}r^{d-1}dr} \\
	\Leftrightarrow & S_d = \frac{2\pi ^{d/2}}{\int_{0}^{\infty}e^{-r^2}r^{d-2}2rdr} \\
	\Leftrightarrow & S_d = \frac{2\pi ^{d/2}}{\int_{0}^{\infty}e^{-r^2}r^{d-2}dr^2} \\
	\Leftrightarrow & S_d = \frac{2\pi ^{d/2}}{\int_{0}^{\infty}e^{-u}u^{\frac{d}{2}-1}du} \\
	\Leftrightarrow & S_d = \frac{2\pi ^{d/2}}{\Gamma(d/2)} (Inserting (1.44))
	\end{aligned}
	\end{equation}
	
	Hence we have proved the (1.43).
	
	When $d = 2$, $S_d = \frac{2\pi}{\Gamma(1)} = 2\pi$ and when $d = 3$ we have $S_d = \frac{2\pi^{3/2}}{\Gamma(3/2)} = 4\pi$, which are equal to well-know expressions of circle and 3D sphere.
	
	\item By utilizing the (1.43)
	\begin{equation}
	\begin{aligned}
	 S_d(r)& = S_d(1)r^d = S_dr^{d-1} \\
	 V_d& = \int_{0}^{a}S_d(r)dr \\
	\Leftrightarrow & = S_d\int_{0}^{a}r^{d-1}dr\\
	\Leftrightarrow & = \frac{S_da^d}{d}\\
	\end{aligned}
	\end{equation}
	It's easy to know that the volume of hypercube is $(2a)^d$, so the ratio would be
	\begin{equation}
	\begin{aligned}
	\frac{VS_d}{VC_d} &= \frac{S_da^d}{d(2a)^d}\\
	&=\frac{2\pi^{d/2}a^d}{d2^da^d\Gamma(d/2)}\\
	&=\frac{\pi^{d/2}}{d2^{d-1}\Gamma(d/2)}\\
	\end{aligned}
	\label{eq:4}
	\end{equation}
	When $d\rightarrow\infty$, we could use Stirling approx(1.47) to derive that
	\begin{equation}
	\begin{aligned}
	 \lim_{d\rightarrow\infty}  \Gamma(x + 1) &\approx (2\pi)^{1/2}e^{-x}(x)^{x+1/2}\\
	 &\textgreater e^{-x} x^x \\
	 & =(\frac{x}{e})^x \rightarrow \infty\\
	\end{aligned}
	\label{eq:5}
	\end{equation}
	Right now we now when d is large enough the $\Gamma$function would be close to infinity, for the rest part of the eq\ref{eq:4} we have
	\begin{equation}
	\lim_{d\rightarrow\infty}\frac{\pi^{d/2}}{d2^{d-1}} \textless \frac{\pi^{d/2}}{2^d}=(\frac{\pi}{4})^{d/2}\rightarrow 0
	\label{eq:6}
	\end{equation}
	By combining eq\ref{eq:5} and eq\ref{eq:6} we can draw a conclution that
	\begin{equation}
	\lim_{d\rightarrow \infty}\frac{VS_d}{VC_d} \rightarrow \frac{0}{\infty} \rightarrow 0
	\end{equation}
	The distance from center to corner is $\sqrt{\Sigma_{i=1}^{d}a^2}=\sqrt{d}a$ while the height(center to hypersurface distance) is simply a, therefore the ratio is $\sqrt{d}$, and when d is large enough this ratio goes to infinity.
	\item
	
	\item
\end{enumerate}

%----------------------------------------------------------------------------------------

\section*{Question 2}

\begin{problem}
	Logistic Regression
\end{problem}

%------------------------------------------------

\subsection*{Answer}



%----------------------------------------------------------------------------------------


\end{document}
